\documentclass[12pt]{article}
\begin{document}
\title{Lab2 - Q-Learning}
\author{Maxime Lhoustau and Philipp Schlieker}
\maketitle

%\section*{General Definitions}
%$M = (S,A,T,R)$ with $S$: States, $A$: Actions, $T=P(S_{t+1} \mid S_t, A_t)$: Transition Probability, $R$: Reward\\
%Markov Property: The transition proabilities depend only on the current state and not on the history of predecessor states.

\section*{Question 1}
%A policy ($\pi$) is a complete mapping from states to actions.\\
%policy: States $\rightarrow$ Actions 
\begin{enumerate}
\item $\pi(S_0) = a_1$
\item $\pi(S_0) = a_2$
\item $\pi(S_1) = a_0$
\item $\pi(S_2) = a_0$
\item $\pi(S_3) = a_0$ 
\end{enumerate}
\section*{Question 2}
\begin{itemize}
\item $V^{*}(S_0) = R(S_0)+max_a \gamma \sum_{S'}T(S_0,a,S')V^{*}(S') =$\\$0 + max(\gamma T(S_0, a_1, S_1)V^{*}(S_1), \gamma T(S_0, a_2, S_2)V^{*}(S_2)) =$\\$ max(\gamma V^{*}(S_1), \gamma V^{*}(S_2))$
\item $V^{*}(S_1) = R(S_1)+max_a \gamma \sum_{S'}T(S_1,a,S')V^{*}(S') =$\\
$0 + max(\gamma (T(S_1, a_0, S_1)V^{*}(S_1) + T(S_1, a_0, S_3)V^{*}(S_3)) =$\\
$\gamma ((1-x)V^{*}(S_1) + xV^{*}(S_3))$
\item $V^{*}(S_2) = R(S_2)+max_a \gamma \sum_{S'}T(S_2,a,S')V^{*}(S') =$\\
$1 + max(\gamma (T(S_2, a_0, S_0)V^{*}(S_0) + T(S_2, a_0, S_3)V^{*}(S_3))) =$\\
$1 + \gamma ((1-y)V^{*}(S_0) + yV^{*}(S_3))$
\item $V^{*}(S_3) = R(S_3)+max_a \gamma \sum_{S'}T(S_3,a,S')V^{*}(S') =$\\
$10 + max(\gamma T(S_3, a_0, S_0)V^{*}(S_0)) =$\\
$10 + \gamma S_0$
\end{itemize}
\section*{Question 3}
Let $x=0$ then $V^{*}(S_1) = 0 + \gamma V^{*}(S_1) = 0$ $\Rightarrow$ $\forall y \in [0,1] : V^{*}(S_1) < V^{*}(S_2) \Rightarrow \pi^{*}(S_0) = a_1$
\section*{Question 4}
$\not \exists y \in [0,1]$ s.t. $\forall \; V \in [0,1] : \pi^{*}(S_0) = a_0$.\\
Proof by contradiction: If such y exists for which $V^{*}(S_1) > V^{*}(S_2)$, then $V^{*}(S_1) > V^{*}(S_2)$ must also hold for the $y$ that minimizes $V^{*}(S_2)$.\\ 
$arg_y min V^{*}(S_2) = 0$. Let $y=0, x=0.00001$ then $V^{*}(S_1) < V^{*}(S_2) \Rightarrow \pi^{*}(S_0) = a_2$ \\
Hence no such $y$ can exist.
\section*{Question 5}
$V = [14.18, 15.76, 15.70, 22.76]$\\
$\pi = [1, 0, 0, 0]$
\end{document}
